\begin{abstract}

Software obfuscation is a commonly used technique to protect software 
especially against white-box attacks, where the adversary controls the host on 
which the software runs. It is a form of security through obscurity and is 
commonly used for intellectual property and DRM (Digital Rights Management) 
protection, defense against reverse-engineering attacks, and even used by 
malware. Some of the popular obfuscating techniques, including changing the 
control flow graph and substituting simpler instruction sequences with complex 
instructions, may however make the obfuscated binary more vulnerable to 
Return-Oriented Programming (ROP) based attacks. We analyze the ROP gadgets 
present in both -- obfuscated and un-obfuscated versions of well 
known binaries, to see if obfuscation makes ROP based exploits easier. Based 
on our study, we identify which particular obfuscation technique(s) results in 
more probable ROP attack gadgets. 

\end{abstract}
