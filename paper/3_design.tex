\section{Overview and Experimental Design}
\label{sec:design}

In this paper, we hypothesize that \textit{obfuscation tools make software 
more susceptible to ROP attacks}. This is based on our observation that 
obfuscation tools add redundant code to the software, change control 
flows, add conditionals, etc. which can increase the number of gadgets in 
the binary.

In order to study the susceptibility of a software to ROP attacks, we 
identify the catalog of gadgets in the code. It may seem that a software 
with a larger number of gadgets would be more susceptible to ROP attacks. 
However, there is limited advantage of repetitive gadgets, and variety is 
more important in order to chain them together and accomplish a useful 
task. Hence, if a software contains a full catalog of gadgets that is  
Turing-complete, then we consider it to be more susceptible than a 
software with an incomplete catalog of gadgets. 

To evaluate our hypothesis, we find and compare gadgets in obfuscated and 
unobfuscated versions of a large number of binaries. We use a set of open 
source software of different size and type including libraries. We use 
Linux as both the development and the target platform for the binaries, 
and use LLVM tool chain for compiler. This is in order to use an LLVM 
based obfuscation tool called Obfuscator-LLVM~\cite{webLLVMobfusc}. The 
tool performs obfuscation transformations on LLVM bytecode, and thus can 
work with a variety of LLVM front-end (e.g. C, C++, etc.) and back-ends 
(x86, MIPS, ARM, etc.). We build the obfuscated version of the software 
for Linux using the same clang compiler with the same flags as used to 
build the unobfuscated binaries. We perform our analysis on both these 
sets of binaries. 

We use an ROP gadget finding tool, ROPgadget~\cite{webRopgadget}, to 
identify potential gadgets in binary. A list of all unique gadgets found 
in the unobfuscated binary is created, and then compared against the list 
from obfuscated binary. We have written our own tool to categorize these 
gadgets based on their functionality. We also look at the total number of 
gadgets found, and the total number of unique gadgets found in a binary.


\subsection{Binaries Selection}
\label{sec:sampleselect}
We use two sets of software, GNU Coreutils \cite{coreutilsdd} and 
OpenSSL~\cite{webOpenssl} as target binaries for our evaluation. Since we 
believe that the increase in size of binary due to obfuscation may impact 
its ROP susceptibility, we want to evaluate on binaries of different 
sizes. In order to be realistic, we should use binaries that are likely to 
be obfuscated. A DRM protection software is a prime candidate for this, 
however, such software is unlikely to be distributed as open source. We 
decided instead to use software that is security sensitive but openly 
available. We use OpenSSL, and specifically its libraries lissl and 
libcrypto as an example of commonly used security sensitive libraries. We 
tried to use glibc, which is more commonly used but it does not compile 
under LLVM toolchain. 

GNU Coreutils include utilities like \texttt{cp, mv, ls, date}, etc. which 
are included in nearly all Linux distributions. Some of these utilities 
can also be considered to be part of these Linux distribution's trusted 
computing base (TCB) since they get frequently run as root. The Coreutils 
package includes 106 utilities of varying size and thus provides us with a 
good base for our evaluation. For more detailed analysis we select a 
subset of these utilities again based on their security sensitiveness, and 
their potential to do harm. These include \texttt{link, chroot, shred, 
touch, date, cp}. We select \texttt{touch} and \texttt{date} due to recent 
vulnerability CVE-2014-9471 that may allow arbitrary code execution or 
denial of service attack. We select \texttt{shred} due to expectation of 
secure removal of files, while the others (\texttt{link, cp, chroot}) for 
their potential for misuse.


\subsection{Obfuscation Tool}
\label{sec:obfllvm}
While there are several software obfuscation tools available, we had 
difficulty finding suitable tools for our purpose. Software obfuscation is 
particularly popular for Java programs, since they are easy to reverse 
engineer, and a large number of tools and techniques are designed for 
Java. Majority of obfuscation tools for C/C++ are neither open-source nor 
free. The Tigress~\cite{webTigress} is one of the free tools available for 
C, with many obfuscation transforms. However, it is not open source and 
does not seem to be actively maintained. 
Obfuscator-LLVM~\cite{webLLVMobfusc} is an open source project to build an 
obfuscator for LLVM tool-chain. It works on LLVM's intermediate 
representation (IR) level, and so it can take advantage of LLVM's 
front-end and back-end which support many languages including C and C++, 
and architectures such as x86, arm, mips, etc. Since, this is a fairly new 
project, it has limited obfuscating transforms available. These are:

\begin{itemize}
    \item Instruction Substitution (SUB): This obfuscation technique 
    relies on substituting one set of instructions with another set of 
    instructions while maintaining the same functionality. In 
    Obfuscator-LLVM, this obfuscation substitutes standard binary 
    operators (like addition, subtraction or boolean operators) by 
    functionally equivalent, but more complicated sequences of 
    instructions. This is a basic, straightforward obfuscation and does 
    not add a lot of security. 
    
    \item Control Flow Flattening (FLA): This obfuscation flattens the 
    control flow graph of the program so that the structure of the program 
    cannot be easily understood by static analysis like disassembly. This 
    is achieved by identifying and moving blocks in a function which are 
    at nested levels, next to each other. The selection of control flow to 
    a particular block is done using a switch statement and a control 
    variable that keeps track of the state of the program. 

    \item Bogus Control Flow (BCF): This obfuscation modifies a function 
    call graph by adding a basic block before the current basic block. 
    This new basic block contains an opaque predicate and then makes a 
    conditional jump to the original basic block.
\end{itemize}
