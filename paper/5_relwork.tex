\section{Related Work}
\label{sec:relwork}

Studying the effects of software obfuscation in increasing the changes for ROP 
exploits is new and no previous work has been done in the areas involving both 
ROP exploitation and code obfuscation. 

Pappas et. el, has proposed various ROP defenses: (1) kBouncer 
\cite{pappas2013transparent}, a light weight tool to prevent certain ROP 
attacks using hardware features and without requiring any modifications of the 
binary code; (2) \cite{pappas2012smashing}, showed, selective type of gadgets 
could be prevented using narrow code transformations of intended gadget 
instruction sequences. Another hardware based approach, ROPecker 
\cite{cheng2014ropecker}, prevents control flow based ROP attacks by examining 
the last branch taken registers, found in commodity processors. Abadi et. al., 
\cite{abadi2005control} used control-flow integrity (CFI) to prevent stack 
based ROP attacks. Carilini et. al., \cite{carlini2014rop} discuss three new 
ROP attacks that breaks existing CFI based defense mechanisms such as kBouncer 
and ROPecker. Schuster et. al., analyze different defense mechanisms in 
\cite{schuster2014evaluating} and show that with a little extra effort, it is 
possible to break ROPecker, kBouncer and ROPGuard.

In~\cite{lu2014ropsteg}, the return-oriented programming is used for program 
steganography. This is a form of software obfuscation using return-oriented 
programming. 
