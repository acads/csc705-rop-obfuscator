\section{Conclusion}
\label{sec:conc}

Software developers concerned about reverse-engineering attacks and piracy 
commonly deploy software obfuscation as a defense. The users of software 
however are more concerned about vulnerabilities in the software that may 
compromise their system. We show that there is a possibility of conflict 
between these two security goals if software obfuscation is used. We have 
shown that software obfuscation significantly increases (1.5x to 3x) the 
number of gadgets in a binary. We have also shown that for logic and control 
flow related gadgets, the increase is much higher (up to 6x). For certain, 
especially smaller, binaries which have very small number of logic and control 
gadgets this increase can potentially make ROP attacks feasible, or at best, 
easier. 
