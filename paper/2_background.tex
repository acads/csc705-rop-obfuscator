\section{Background}
\label{sec:background}

\subsection {Return Oriented Programming}
Return Oriented Programming \cite{roemer2012return} is made possible by 
making use of gadgets. Each gadget ends with a `return' or `jump' 
instruction, which is used to chain together several such gadgets to alter 
the program's behavior and each gadget might perform a different operation, 
say a load, adding registers and a jump. On the whole, return oriented 
programming can be viewed as the generalized notion of return-to-libc 
\cite{wojtczuk2001advanced} types of attacks. Current research is focused on 
defending, or eliminating, return oriented programming attacks as they are 
proven \cite{krahmer2005x86} to be ineffective against the major defense 
against execution of malicious code - W$\oplus$X protection model. In 
W$\oplus$X model, a memory page is either writable or executable, but not 
both, which prevents all types of code injection attacks. In return oriented 
programming attacks, the attacker does not inject any code and just alters the 
program execution by executing already existing code in an arbitrary fashion.

A return oriented program consists of several gadgets arranged carefully to 
meet the attacker's goal. These gadgets must be in the memory, in the address 
space of the executing program or in the address space of a library used by 
the program. Most of the gadgets fall under five broad categories: (1) load/
store (move to/from register/memory), (2) arithmetic and logical (add, sub, 
and, xor, bitwise and shifts), (3) control flow (jmp and ret), (4) system 
calls (int 0x80 followed by call, ret) and (5) function calls (call, ret). The 
instruction sequences that are chosen to be gadgets can be identified by the 
presence of `ret' instructions or by identifying unintended instruction 
sequences. The latter is possible only in case of x86 architecture as it 
supports unaligned, variable length instructions. This can be exploited by 
moving the instruction pointer to an offset within the instruction, which 
changes the interpretation of the current and the following instructions until 
a `ret' or a `jmp' instruction is decoded. The gadgets need not to be 
contiguous and can be chained by means of return or jump instructions. The 
first in line gadget should be executed by redirecting the stack pointer, 
which can be accomplished by a simple buffer overflow.


\subsection {Defenses Against ROP}
The attacks that fall under return oriented programming paradigm are very 
broad, but still there are many defense mechanisms to mitigate or to prevent 
ROP based attacks. This section briefs few such mechanisms.

Code randomization \cite{pappas2012smashing} is a technique in which the 
intended instruction sequences, of a binary, to be used as a gadget, are 
randomized through in-place, narrow code transformations. The performance 
overhead is negligible as no new code is inserted. When used along with other 
code randomization procedures, such as, Address Space Layout Randomization 
(ASLR) \cite{shacham2004effectiveness}, it disrupts 10\% and 80\%, effectively 
and probabilistically, of the instruction sequences in a binary. 
ROPdefender \cite{davi2011ropdefender} is a tool that adds additional code to a 
binary, to observe the program's execution behavior, using just-in-time 
compiler. It can prevent all return instruction based attacks and detect 
unintended instruction attacks. It doesn't need the program source code as it 
uses a JIT compiler. The performance overhead of ROPdefender is almost 100\%. 
XFI \cite{erlingsson2006xfi}, based on control flow integrity 
\cite{abadi2005control}, in addition to flexible access control and 
fundamental integrity guarantees, also offers stack protection with return 
address safety. XFI can protect return address based ROP attacks using 
intended instruction sequences.


\subsection{Software Obfuscation Techniques}
Software obfuscation is achieved by a sequence of transformations on the code 
in order to obscure the purpose of the code while maintaining its original 
behavior. There have been several obfuscating transformations identified in 
literature ~\cite{nagra2009surreptitious, collberg2002watermarking}. These  
include inserting opaque predicates which are difficult to evaluate at compile 
time; inserting dead or irrelevant code; cloning, splitting or merging 
functions; control flow flattening; etc. More details about some of these 
obfuscation techniques are discussed in section~\ref{sec:obfllvm}.
