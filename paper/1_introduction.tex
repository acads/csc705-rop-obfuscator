\section{Introduction}
\label{sec:introduction}

Software developers have two broad security concerns: (1) vulnerabilities that 
lead to exploitation of the software (e.g. buffer overflow); (2) reverse 
engineering (e.g. software or music piracy). The first concern is based on the 
threat model where the software runs on a benign host, and the goal is to 
protect the software in order to protect the host system and the information 
the software has access to. An example of this threat model is an unprivileged 
attacker exploits vulnerability in software to gain access to restricted 
information. The second concern however, is based on a very different threat 
model where the host or the privileged user is malicious and tries subvert the 
restrictions set by software developers. An example would be reverse 
engineering DRM protection software for music or movies piracy. 

Some developers use static analysis tools to prevent vulnerabilities such as 
buffer overflow; however, many others rely on OS level protections such as 
ASLR and W$\oplus$X. Protection against reverse engineering is generally 
achieved via software obfuscation tools and cannot leverage OS support. 

Given the pervasiveness of ASLR and W$\oplus$X in operating systems, attackers 
have begun using return oriented programming. Return Oriented Programming 
(ROP) is a technique through which an attacker can introduce changes to a 
program's control flow using many short code snippets,  
called gadgets, present in a program's address space. ROP allows attackers to  
exploit buffer overflow vulnerabilities even when ASLR or W$\oplus$X 
protection is enabled. The goal of the attacker using ROP is to exploit a 
vulnerability (buffer overflow) in client software in order to compromise the 
host system. Thus, this is a technique used by attackers in our first threat 
model. 

Even the software that requires protection against the second threat model 
should be secure against the first threat model in order to protect the benign 
hosts it may run on. However, some of the software obfuscation techniques used 
in practice may impact the vulnerability of the software to the first threat 
model. Some commonly used obfuscation techniques include adding dead or 
irrelevant code, and changing control flow graphs by splitting or merging 
functions and adding opaque predicates. We study the impact of software 
obfuscation on the program's vulnerability to ROP based attacks. An ROP based 
attack depends on stringing together gadgets in the code in order to perform 
arbitrary action. In ~\cite{roemer2012return}, a catalog of x86 gadgets are 
identified for a Turing-complete ROP. A software with a larger number of 
gadgets from this catalog is more easier to exploit using ROP by an attacker, 
assuming they find an entry through vulnerabilities like buffer overflow. We 
analyze a set of open source software binaries in terms of the catalog of 
gadgets they contain, and compare this with the gadgets found in obfuscated 
versions of the same software. Based 
on our experiments, we show that obfuscation significantly impacts the number of gadgets in a binary and can potentially make an ROP attack easier. 

The rest of the paper is organized as follows. Section 2 explains the 
background information for this work, including a brief introduction to 
return oriented programming, current state of defenses against ROP and about 
popular code obfuscation techniques. Section 3 provides a brief overview of 
our work, outlines the design and implementation details, followed 
by evaluation methods and results in section 4. Section 5 includes related work and finally we conclude our work in section 6.
